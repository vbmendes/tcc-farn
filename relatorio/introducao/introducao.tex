%%
%% Captulo 1: Modelo de Captulo
%%

% Est sendo usando o comando \mychapter, que foi definido no arquivo
% comandos.tex. Este comando \mychapter  essencialmente o mesmo que o
% comando \chapter, com a diferena que acrescenta um \thispagestyle{empty}
% aps o \chapter. Isto é necessário para corrigir um erro de LaTeX, que
% coloca um número de pgina no rodapé de todas as páginas iniciais dos
% capítulos, mesmo quando o estilo de numeração escolhido é outro.
\mychapter{Introdução}
\label{Cap:introducao}

Este trabalho trata de um cliente de \textit{Web Service} SOAP que deve obter o mais rapidamente possível uma grande quantidade de dados do servidor e persistí-los em seu banco de dados. Existem inúmeras formas de resolver esse problema. Neste trabalho serão abordadas algumas técnicas possíveis.

\section{Descrição geral}

O trabalho é baseado em um \textit{Web Service} SOAP, que é um protocolo para troca de informações estruturadas em uma plataforma descentralizada e distribuída.

Um portal de internet precisa oferecer a seus usuários em tempo real dados que são obtidos através de um servidor que os fornece através de um \textit{web service}. Para não ter atraso na informação oferecida, ele precisa requisitar ao fornecedor de dados o mais frequentemente possível ao mesmo tempo que tem que manter um bom tempo de resposta para seus usuários.

\section{Objetivo}

O objetivo do trabalho é resolver os problemas relacionados à otimização do tempo de resposta e à frequência de atualização do dado. Por se tratar de um problema de otimização de tempo de resposta é necessário pensar na latência e no \textit{overhead} gerado por uma conexão HTTP e analisar a melhor arquitetura de sistema possível.

\section{Motivação}

Com o avanço da informática e de conceitos como a computação em nuvem, cada vez mais os servidores estão em constante comunicação ao redor do mundo. Seja para autenticar usuários, seja para persistir dados ou seja para solicitar dados. Um portal de notícias, por exemplo, exibe, além das notícias, dados meteorológicos e econômicos. Esses dados normalmente não são produzidos pelo próprio portal e então são necessárias parcerias com fornecedores de conteúdo que disponham de tal informação. Estes fornecedores, em geral, disponibilizam os dados através de um \textit{Web Service}.

Dado que no mundo jornalístico a agilidade na entrega de informações é um fator primordial para o sucesso, surge a necessidade de otimizar o tempo de consumo de tais fornecedores a fim de entregar aos usuários a informação mais recente possível.

\section{Metodologia}

O projeto foi dividido em cinco partes: 

\begin{itemize}
\item estudo sobre \textit{web services} e o protocolo SOAP;
\item estudo sobre computação paralela;
\item implementação de cliente para \textit{web services} SOAP;
\item implementação de estratégias de utilização do cliente.
\end{itemize}


\subsection{Estudo sobre \textit{web services} e o protocolo SOAP}

Por se tratar de atualização de \textit{web services} se faz necessário um estudo sobre os conceitos desta tecnologia.

\subsection{Estudo sobre computação paralela}

O projeto tem como pré-requisito oferecer um baixo tempo de resposta. A computação paralela é um paradigma muito interessante para otimizar tarefas computacionais e explorar a máquina o máximo possível.

\subsection{Implementação de cliente para \textit{web service} SOAP}

Depois de feito o estudo, foi colocado em prática os conhecimentos a fim de implementar uma solução que possibilitasse o acesso a \textit{web services} que utilizem o protocolo SOAP.

\subsection{Implementação de estratégias para utilização do cliente}

Com uma forma de acessar os dados, foram elaboradas duas estratégias para otimizar o tempo de resposta ao usuário final e garantir a recência dos dados apresentados.

