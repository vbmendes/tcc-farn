\mychapter{Pontos de melhoria}
\label{Cap:pontos_de_melhoria}

Para otimizar ainda mais a performance da atualiza��o dos dados, � poss�vel, entre outros, melhorar a infraestrutura ou priorizar atualiza��o das cota��es mais importantes.

\section{Melhorias de infraestrutura}

Existem tr�s grandes vari�veis na infraestrutura do atualizador das cota��es. S�o elas::

\begin{itemize}
\item O(s) servidor(es) que fornece(m) o \textit{web service}, que no caso n�o est�(�o) sob controle;
\item O(s) servidor(es) que rodar�(�o) os processos de atualiza��o;
\item A conex�o entre os servidores.
\end{itemize}

A conex�o agilizar� a troca de mensagens entre os servidores, mas de nada adianta ter uma velocidade superdimensionada e servidores com baixa capacidade de atendimento.

Um ponto a ser testado � distribui��o da atualiza��o em mais de um servidor, possibilitando que a aplica��o escale horizontalmente. Neste caso, o servidor da fonte de dados pode ficar sobrecarregado.

\section{Priorizar atualiza��o de cota��es}

No mercado de cota��es sempre existem pap�is mais relevantes que outros. A cota��o do D�lar Americano � consideravelmente mais relevante que a do D�lar do Zimbabwe. Portanto, � poss�vel estabelecer prioridades entre as cota��es de forma que um grupo seleto delas seja atualizado mais frequentemente.
