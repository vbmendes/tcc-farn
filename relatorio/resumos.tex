%
% ********** Resumo
%

% Usa-se \chapter*, e não \chapter, porque este "capítulo" não deve
% ser numerado.
% Na maioria das vezes, ao invés dos comandos LaTeX \chapter e \chapter*,
% deve-se usar as nossas versões definidas no arquivo comandos.tex,
% \mychapter e \mychapterast. Isto porque os comandos LaTeX têm um erro
% que faz com que eles sempre coloquem o número da página no rodapé na
% primeira página do capítulo, mesmo que o estilo que estejamos usando
% para numeração seja outro.
\mychapterast{Resumo}

Neste trabalho será relatado o desenvolvimento de um cliente de \textit{Web Service} baseado em SOAP que deve realizar várias requisições para sincronizar uma grande massa de dados em um tempo relativamente curto. Foram abordadas três técnicas: requisições síncronas e sequenciais em um único processo; requisições síncronas em vários processos; e requisições assíncronas que se beneficiam do uso de I/O não bloqueante.

\vspace{1.5ex}

{\bf Palavras-chave}: \textit{Web services}, I/O não bloqueante, SOAP, computação distribuída, programação concorrente.
%
% ********** Abstract
%

\mychapterast{Abstract}

On the following work, will be reported the development of a Web Service client based on SOAP that should make many requests to synchronize a huge data in a short time. Three techniques were considered: synchronous and sequential requests in a single process; synchronous requests with multiprocessing; and asynchronous request that take advantage of non-blocking I/O.

\vspace{1.5ex}

{\bf Keywords}: Web Services, non-blocking I/O, SOAP, distributed computing, multiprocessing.
