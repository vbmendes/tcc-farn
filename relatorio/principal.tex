%%
%% Arquivo principal para:
%% - teses de doutorado
%% - dissertações de mestrado
%% - exames de qualificao de mestrado e doutorado
%%
%% NOTA: A PUBLICAÇÃO DESTE MODELO VISA APENAS ORIENTAR OS PÓS-GRADUANDOS
%% NA PREPARAÇÃO DE SEUS TEXTOS. O PPgEE DA UFRN NÃO PROVÊ ASSISTÊNCIA NO
%% USO DAS FERRAMENTAS NECESSÁRIAS AO USO DESTE MODELO (LATEX, XFIG, ETC.)
%%
%% Adelardo Medeiros, dezembro de 2005.

% DEFINIÇÕES GLOBAIS

% Esta primeira linha dá informações gerais sobre o documento.
% PARA A VERSÃO FINAL:
% papel A4, letra grande (12pt), openright (capítulos só iniciam em
% página direita, se necessário incluindo uma pgina em branco),
% twoside (o documento vai ser impresso em frente e costa)
\documentclass[a4paper,12pt,openright,twoside]{book}
% PARA A QUALIFICAÇÃO E PARA A VERSÃO INICIAL:
% papel A4, letra grande (12pt), openany (capítulos iniciam em
% qualquer página), oneside (o documento vai ser impresso só na frente)
%\documentclass[a4paper,12pt,openany,oneside]{book}

% PACOTES OBRIGATÓRIOS

% Use estes pacotes para poder digitar diretamente as letras com acento
% e para que a hifenização funcione corretamente
\usepackage[utf8]{inputenc}
\usepackage{ae}
% Para usar fontes standard ao invés das do LaTeX (gera melhores PDFs)
\usepackage{pslatex}
% Para a hifenização em português
\usepackage[portuges, brazil]{babel}
% Para que os primeiros parágrafos das seções também sejam indentados
\usepackage{indentfirst}
% Para poder incluir gráficos (figuras)
\usepackage{graphicx}
% Para poder fazer glossário ou lista de símbolos
% Use a segunda opção se quiser incluir na definição do símbolo a
% página e/ou a equação onde ela foi definida
% \usepackage[portuguese,noprefix]{nomencl}
%\usepackage[portuguese,noprefix,refeq,refpage]{nomencl}
% Para permitir espaçamento simples, 1 1/2 e duplo
\usepackage{setspace}
% Para usar alguns comandos matemáticos avançados muito úteis
\usepackage{amsmath}
% Para poder usar o ambiente "comment"
\usepackage{verbatim}
% Para poder ter tabelas com colunas de largura auto-ajustável
\usepackage{tabularx}
% Para executar um comando depois do fim da página corrente
\usepackage{afterpage}
% Para formatar URLs (endereços da Web)
\usepackage{url}
% Para reduzir os espaços entre os ítens (itemize, enumerate, etc.)
% Este pacote não faz parte da distribuição padrão do LaTeX.
\usepackage{noitemsep}
% Para as citações bibliográficas
\usepackage[abbr]{harvard}	% As chamadas são sempre abreviadas
\harvardparenthesis{square}	% Colchetes nas chamadas
\harvardyearparenthesis{round}	% Parêntesis nos anos das referências
%\renewcommand{\harvardand}{e}	% Substituir "&" por "e" nas referências

% PACOTES OPCIONAIS

% Para poder incluir arquivos Postscript com cores (do Xfig, por exemplo)
\usepackage{color}
% Para ter células em tabelas que ocupam mais de uma linha
\usepackage{multirow}
% Para poder ter tabelas longas em mais de uma página
%\usepackage{longtable}
% Para poder escrever partes do texto em "n" colunas
%\usepackage{multicol}
% Se você quiser personalizar os cabeçalhos das páginas
%\usepackage{fancyheadings}
% Para incluir algoritmos e listagens de códigos
%\usepackage{listings}
% Capítulos com títulos em um formato "decorado"
\usepackage{capitulos}

% As definiçõoes dos novos comandos estão agrupadas no arquivo "comandos.tex"
% Esta separação é opcional: se você preferir, pode por as definicoes
% diretamente neste arquivo
\input{comandos.tex}

%
% As margens
%

% Direcao horizontal

% Margem interna
% Nas paginas impares
\setlength{\oddsidemargin}{3.5cm}         % Margem real desejada
% Nas paginas pares
\setlength{\evensidemargin}{2.5cm}        % Margem real desejada
% Largura do texto
\setlength{\textwidth}{15cm}
% As margens laterais no LaTeX sao sempre 1 polegada maiores do que as
% fixadas. Se foi fixada \setlength{\oddsidemargin}{3.5cm}, a margem
% real seria de 3.5+2.54=6.04cm. Para permitir que voce nao tenha que
% fazer esta conta, pode usar o numero desejado nas linhas anteriores
% e a gente subtrai 1in nas proximas linhas
\addtolength{\oddsidemargin}{-1in}
\addtolength{\evensidemargin}{-1in}
% Note que a margem direita nao e fixada diretamente:
% ela e obtida subtraindo-se os outros valores da largura da pagina.
% 3.5+15+x=21cm (largura A4) -> x = margem externa = 2.5cm

% Direcao vertical

% Margem superior (entre o topo da folha e o cabecalho), altura do
% cabecalho e distancia entre o fim do cabecalho e o inicio do texto
\setlength{\topmargin}{2.0cm}             % Margem real desejada
\setlength{\headheight}{1.0cm}
\setlength{\headsep}{1.0cm}
% Altura do texto (sem cabecalho e rodape)
\setlength{\textheight}{22.7cm}
% Distancia do fim do texto ao rodape
\setlength{\footskip}{1.0cm}
% A margem superior no LaTeX e sempre 1 polegada maior do que a
% fixada. Se foi fixada \setlength{\topmargin}{2.0cm}, a margem
%real seria de 2.0+2.54=4.54cm. Para permitir que voce nao tenha que
% fazer esta conta, pode usar o numero desejado na linha anterior
% e a gente subtrai 1in na proxima linha
\addtolength{\topmargin}{-1in}
% Note que a margem inferior nao e fixada diretamente:
% ela e obtida subtraindo-se os outros valores, sem incluir o
% "footskip", da altura da pagina.
% 2.0+1.0+1.0+22.7+x=29.7cm (altura A4) -> x = margem inferior = 3cm

%
% O estilo das referencias bibliograficas
%

\bibliographystyle{ppgee}

%
% O espacamento entre linhas
%

% As paginas iniciais sao sempre em espacamento simples
\singlespacing

% Para a criacao do glossario (ou lista de simbolos)
\makeglossary

% Lista de arquivos a serem processados. Estes comandos "includeonly" sao
% dispensaveis e devem obrigatoriamente ser comentados na hora de gerar o
% documento definitivo. Eles servem para compilar apenas parte do documento.
% e util, durante a redacao, para que nao se tenha de compilar todo o
% documento a cada vez que se faz uma alteracao. Por exemplo, se eu estou
% fazendo alteracoes na dedicatoria e as outras partes nao tem interesse no
% momento, posso incluir (descomentar) a linha "\includeonly{preambulo}"
%\includeonly{rosto}
%\includeonly{catalograficos}
%\includeonly{preambulo}
%\includeonly{resumos}
%\includeonly{introducao/introducao}
%\includeonly{estilo/estilo}
%\includeonly{matematica/matematica}
%\includeonly{figuras/figuras}
%\includeonly{conclusao/conclusao}
%\includeonly{apendice/apendice}

% Inicia o texto
\begin{document}

% Paginas iniciais (sem numeracao)
\pagestyle{empty}

% Pgina de rosto (capa interna)
%
% ********** Página de Rosto
%

% titlepage gera páginas sem numeração
\begin{titlepage}

\begin{center}

\small

% O comando @{} no ambiente tabular x é para criar um novo delimitador
% entre colunas que não a barra vertical | que é normalmente utilizada.
% O delimitador desejado vai entre as chaves. No exemplo, não há nada,
% de modo que o delimitador é vazio. Este recurso está sendo usado para
% eliminar o espaço que geralmente existe entre as colunas
% \begin{tabularx}{\linewidth}{@{}l@{}C@{}r@{}}
% A figura foi colocada dentro de um parbox para que fique verticalmente
% centralizada em relação ao resto da linha
% \parbox[c]{3cm}{\includegraphics[width=\linewidth]{FARN}} &
\begin{center}
\textsf{\textsc{Faculdade Natalense para o Desenvolvimento do Rio Grande do Norte}}
\end{center} % &
%% \parbox[c]{2cm}{\includegraphics[width=\linewidth]{figuras/dca}}
% \end{tabularx}

% O vfill é um espação vertical que assume a máxima dimensão possível
% Os vfill's desta página foram utilizados para que o texto ocupe
% toda a folha
\vfill

\LARGE

\textbf{Atualização de \textit{Web Services} em Tempo Real}

\vfill

\Large

\textbf{Vinícius Brandão Mendes}

\vfill

\normalsize

Orientador: Prof. Ricardo Wendel
% Se não houver co-orientador, comente a próxima linha
% \\[2ex] Co-orientador: Prof. Dr. Beltrano Catandura do Amaral

\vfill

\hfill
\parbox{0.5\linewidth}{\textbf{%
% Descomente as opçõs que se aplicam ao seu caso
%Proposta de Tema para Qualificação}
Relatório técnico}
%Tese de Doutorado}
apresentado ao Curso Especialização em Desenvolvimento
de Sistemas Corporativos da FARN
como trabalho de conclusão de curso.}
%Doutor em Ciências.}

\vfill

\large

% Este número de ordem deve ser obtido na coordenação do PPgEE
% Corresponde ao número sequencial da sua tese ou dissertação:
% por exemplo, a 25 tese de doutorado terá o número de ordem D25
% Evidentemente, este dado não existe para propostas de tema, caso
% em que a próxima linha deve ser comentada.
%Número de ordem PPgEE: M000

Natal, RN, abril de 2011

\end{center}

\end{titlepage}


% Ficha catalografica: os dados catalograficos devem ser fornecidos
% pela BCZM.
% So sao incluidos na versao final da tese ou dissertacao. Nao sao
% incluidos nem na proposta de tema de qualificacao nem na versao
% preliminar da tese ou dissertacao: nestes casos, comente a proxima linha.
%\include{catalograficos}

% Assinaturas da banca, dedicatoria e agradecimentos
% So sao incluidos na versao final da tese ou dissertacao. Nao sao
% incluidos nem na proposta de tema de qualificacao nem na versao
% preliminar da tese ou dissertacao: nestes casos, comente a proxima linha.
%
% ********** Página de assinaturas
%

\begin{titlepage}

\begin{center}

\LARGE

\textbf{Atualização de \textit{Web Services} em Tempo Real}

\vfill

\Large

\textbf{Vinícius Brandão Mendes}

\end{center}

\vfill

% O \noindent é para eliminar a tabulação inicial que normalmente é
% colocada na primeira frase dos parágrafos
\noindent
% Descomente a opção que se aplica ao seu caso
% Note que propostas de tema de qualificação nunca têm preâmbulo.
Relatório técnico
%Tese de Doutorado
aprovado em 25 de abril de 2011 pela banca examinadora composta
pelos seguintes membros:

% Os nomes dos membros da banca examinadora devem ser listados
% na seguinte ordem: orientador, co-orientador (caso haja),
% examinadores externos, examinadores internos. Dentro de uma mesma
% categoria, por ordem alfab?tica

\begin{center}

\vspace{1.5cm}\rule{0.95\linewidth}{1pt}
\parbox{0.9\linewidth}{%
Prof. Ricardo Wendel (orientador) \dotfill\ DCA/UFRN}

\vspace{1.5cm}\rule{0.95\linewidth}{1pt}
\parbox{0.9\linewidth}{%
Prof. Dr. Pablo Javier Alsina \dotfill\ DCA/UFRN}

\vspace{1.5cm}\rule{0.95\linewidth}{1pt}
\parbox{0.9\linewidth}{%
M. Sc. André Macedo Santana \dotfill\ DCA/UFRN}

\end{center}

\end{titlepage}

%
% ********** Dedicatória
%

% A dedicatória não é obrigatória. Se você tem alguém ou algo que teve
% uma importância fundamental ao longo do seu curso, pode dedicar a ele(a)
% este trabalho. Geralmente não se faz dedicatória a várias pessoas: para
% isso existe a seção de agradecimentos.
% Se não quiser dedicatória, basta excluir o texto entre
% \begin{titlepage} e \end{titlepage}

% \begin{titlepage}

%\vspace*{\fill}

%\hfill
%\begin{minipage}{0.5\linewidth}
%\begin{flushright}
%\large\it
%Aos meus pais pela minha educação, orientação, apoio e atenção dispensados a mim.
%\end{flushright}
%\end{minipage}
%
%\vspace*{\fill}

%\end{titlepage}

%
% ********** Agradecimentos
%

% Os agradecimentos não são obrigatórios. Se existem pessoas que lhe
% ajudaram ao longo do seu curso, pode incluir um agradecimento.
% Se não quiser agradecimentos, basta excluir o texto após \chapter*{...}

\chapter*{Agradecimentos}
\thispagestyle{empty}

\begin{trivlist}  \itemsep 2ex

\item Ao meu orientador, professor Ricardo Wendel, sou grato pela orientação.

\item Aos meus pais, Joalmi Mendes de Oliveira e Maria Gleide Brandão Mendes, pelo apoio total durante a execução do trabalho e durante toda a minha vida, com educação, instrução e companheirismo.

\item A minha noiva Rosana Curvelo de Souza pelo constante incentivo.

\item Ao colega Fábio Miranda Costa pelo apoio durante o curso.

\item Aos meus irmãos, familiares e amigos pela paciência e apoio dispensados durante a execução do trabalho.

\end{trivlist}


%
% O espaamento entre linhas
%

% PARA A VERSO FINAL:
% Deve ser usado espaamento simples nas pginas de texto
\singlespacing
% PARA A QUALIFICAO E PARA A VERSO INICIAL:
% Deve ser usado espaamento 1 1/2 nas pginas de texto
%\onehalfspacing

% Resumo/Abstract
%
% ********** Resumo
%

% Usa-se \chapter*, e não \chapter, porque este "capítulo" não deve
% ser numerado.
% Na maioria das vezes, ao invés dos comandos LaTeX \chapter e \chapter*,
% deve-se usar as nossas versões definidas no arquivo comandos.tex,
% \mychapter e \mychapterast. Isto porque os comandos LaTeX têm um erro
% que faz com que eles sempre coloquem o número da página no rodapé na
% primeira página do capítulo, mesmo que o estilo que estejamos usando
% para numeração seja outro.
\mychapterast{Resumo}

Neste trabalho será relatado o desenvolvimento de um cliente de \textit{Web Service} baseado em SOAP que deve realizar várias requisições para sincronizar uma grande massa de dados em um tempo relativamente curto. Foram abordadas três técnicas: requisições síncronas e sequenciais em um único processo; requisições síncronas em vários processos; e requisições assíncronas que se beneficiam do uso de I/O não bloqueante.

\vspace{1.5ex}

{\bf Palavras-chave}: \textit{Web services}, I/O não bloqueante, SOAP, computação distribuída, programação concorrente.
%
% ********** Abstract
%

\mychapterast{Abstract}

On the following work, will be reported the development of a Web Service client based on SOAP that should make many requests to synchronize a huge data in a short time. Three techniques were considered: synchronous and sequential requests in a single process; synchronous requests with multiprocessing; and asynchronous request that take advantage of non-blocking I/O.

\vspace{1.5ex}

{\bf Keywords}: Web Services, non-blocking I/O, SOAP, distributed computing, multiprocessing.


% Paginas introdutrias (com numerao romana)
\frontmatter

% Lista de contedo (sumrio, gerado automaticamente)
\addcontentsline{toc}{chapter}{Sumário}
\tableofcontents

% Lista de figuras (gerada automaticamente)
\cleardoublepage
\addcontentsline{toc}{chapter}{Lista de Figuras}
\listoffigures

% Lista de tabelas (gerada automaticamente)
%\cleardoublepage
%\addcontentsline{toc}{chapter}{Lista de Tabelas}
%\listoftables

% Glossrio (gerado automaticamente - veja entradas em
% introducao/introducao.tex e em estilo/estilo.tex)
%\cleardoublepage
%\renewcommand{\nomname}{Lista de Smbolos e Abreviaturas}
%\markboth{\MakeUppercase{\nomname}}{\MakeUppercase{\nomname}}
%\addcontentsline{toc}{chapter}{\nomname}
% O argumento opcional do comando \printglossary  a largura deixada
% para os smbolos no glossrio. Se seus smbolos so "largos", como
% neste exemplo,   melhor por mais espao do que o 1cm que  reservado
% por default
% \printglossary[1.5cm]

% Pginas do texto principal (com cabealho)
\mainmatter
\pagestyle{headings}

% Para facilitar a organizao, foi criado um diretrio para cada
% captulo do documento, pois assim os arquivos das figuras ficam
% classificados por captulos

%%
%% Captulo 1: Modelo de Captulo
%%

% Est sendo usando o comando \mychapter, que foi definido no arquivo
% comandos.tex. Este comando \mychapter  essencialmente o mesmo que o
% comando \chapter, com a diferena que acrescenta um \thispagestyle{empty}
% aps o \chapter. Isto é necessário para corrigir um erro de LaTeX, que
% coloca um número de pgina no rodapé de todas as páginas iniciais dos
% capítulos, mesmo quando o estilo de numeração escolhido é outro.
\mychapter{Introdução}
\label{Cap:introducao}

Este trabalho trata de um cliente de \textit{Web Service} SOAP que deve obter o mais rapidamente possível uma grande quantidade de dados do servidor e persistí-los em seu banco de dados. Existem inúmeras formas de resolver esse problema. Neste trabalho serão abordadas algumas técnicas possíveis.

\section{Descrição geral}

O trabalho é baseado em um \textit{Web Service} SOAP, que é um protocolo para troca de informações estruturadas em uma plataforma descentralizada e distribuída.

Um portal de internet precisa oferecer a seus usuários em tempo real dados que são obtidos através de um servidor que os fornece através de um \textit{web service}. Para não ter atraso na informação oferecida, ele precisa requisitar ao fornecedor de dados o mais frequentemente possível ao mesmo tempo que tem que manter um bom tempo de resposta para seus usuários.

\section{Objetivo}

O objetivo do trabalho é resolver os problemas relacionados à otimização do tempo de resposta e à frequência de atualização do dado. Por se tratar de um problema de otimização de tempo de resposta é necessário pensar na latência e no \textit{overhead} gerado por uma conexão HTTP e analisar a melhor arquitetura de sistema possível.

\section{Motivação}

Com o avanço da informática e de conceitos como a computação em nuvem, cada vez mais os servidores estão em constante comunicação ao redor do mundo. Seja para autenticar usuários, seja para persistir dados ou seja para solicitar dados. Um portal de notícias, por exemplo, exibe, além das notícias, dados meteorológicos e econômicos. Esses dados normalmente não são produzidos pelo próprio portal e então são necessárias parcerias com fornecedores de conteúdo que disponham de tal informação. Estes fornecedores, em geral, disponibilizam os dados através de um \textit{Web Service}.

Dado que no mundo jornalístico a agilidade na entrega de informações é um fator primordial para o sucesso, surge a necessidade de otimizar o tempo de consumo de tais fornecedores a fim de entregar aos usuários a informação mais recente possível.

\section{Metodologia}

O projeto foi dividido em cinco partes: 

\begin{itemize}
\item estudo sobre \textit{web services} e o protocolo SOAP;
\item estudo sobre computação paralela;
\item implementação de cliente para \textit{web services} SOAP;
\item implementação de estratégias de utilização do cliente.
\end{itemize}


\subsection{Estudo sobre \textit{web services} e o protocolo SOAP}

Por se tratar de atualização de \textit{web services} se faz necessário um estudo sobre os conceitos desta tecnologia.

\subsection{Estudo sobre computação paralela}

O projeto tem como pré-requisito oferecer um baixo tempo de resposta. A computação paralela é um paradigma muito interessante para otimizar tarefas computacionais e explorar a máquina o máximo possível.

\subsection{Implementação de cliente para \textit{web service} SOAP}

Depois de feito o estudo, foi colocado em prática os conhecimentos a fim de implementar uma solução que possibilitasse o acesso a \textit{web services} que utilizem o protocolo SOAP.

\subsection{Implementação de estratégias para utilização do cliente}

Com uma forma de acessar os dados, foram elaboradas duas estratégias para otimizar o tempo de resposta ao usuário final e garantir a recência dos dados apresentados.



\mychapter{\textit{Web Services}}
\label{Cap:web_services}

\textit{Web service} é uma interface acessível através da rede para uma funcionalidade de uma aplicação construída usando tecnologias definidas como padrão na Internet \cite{ref-oreilly-soap}. É apenas mais uma camada de troca de mensagens entre uma aplicação e outra. A principal vantagem no seu uso é prover comunicação entre diferentes aplicações independente da plataforma ou linguagem de programação utilizada por elas, o que garante interoperabilidade entre sistemas desde que ambos utilizem o mesmo protocolo de comunicação entre si.

Através do uso de \textit{web services} é possível fazer com que uma aplicação faça chamadas de métodos remotos de outra aplicação. É simples como gerar uma requisição que encapsule qual método será chamado e quais os parâmetros e esperar uma resposta com o resultado da chamada do método. Para isso se faz necessário quebrar a camada de aplicação da pilha de camadas do modelo de redes TCP/IP em quatro camadas: aplicação, descoberta, descrição e empacotamento.

\section{Aplicação}

A camada de aplicação é o código que precisará se comunicar com a outra aplicação através do \textit{web service}, que pode ser escrito em qualquer linguagem.

\section{Descoberta}

Esta camada é responsável por disponibilizar metadados sobre \textit{web services} de modo a facilitar a busca pela aplicação necessária para cada projeto.

\section{Descrição}

A descrição define como o \textid{web service} deve ser utilizado. Quais os métodos que ele expõe, quais os parâmetros necessários em cada um deles e quais as possíveis respostas. O \textit{Web Service Description Language} (WSDL) é um padrão muito utilizado para descrever um \textit{web service} através do uso de XML.

\section{Empacotamento}





%% \include{cores/cores}

%% \include{calibracao/calibracao}

%% \include{rotulo/rotulo}

%% \mychapter{Implementação}
\label{Cap:implementacao}

Foram desenvolvidas duas implementações para o problema com o objetivo de encontrar a mais performática:

\begin{itemize}
\item estratégia de atualização de dados sob demanda;
\item atualizador sequencial de dados independente de demanda.
\end{itemize}

\section{Requisições sob demanda}

\section{Atualizar todos os dados de tempos em tempos}

\subsection{Requisições sequenciais}

\subsection{Requisições paralelas com múltiplos processos}



%% \mychapter{Resultados}
\label{Cap:resultados}

Os testes foram feitos em Macbook Pro com processador Intel Core i5 2.3 GHz de 4 n�cleos com 8 GB de mem�ria RAM a 1333 MHz. Como o computador n�o � dedicado apenas aos testes, existem oscila��es referentes a outras tarefas do sistema operacional.

Foram realizados dois testes::

\begin{itemize}
\item Teste de desempenho no atendimento ao usu�rio;
\item Teste de desempenho do atualizador de cota��es.
\end{itemize}

\section{Teste de desempenho no atendimento ao usu�rio}

Foram aferidos o tempo m�dio de resposta (figura~\ref{Fig:grafico_tempos_de_resposta}), o n�mero de usu�rios atendidos por segundo, chamado de \textit{throughput} (figura~\ref{Fig:grafico_throughput}), e o n�mero de requisi��es atendidas em uma simula��o de um usu�rio requisitando repetidas vezes as duas solu��es propostas (atualiza��o de dados sob demanda ao web service e obten��o de dados de um banco de dados atualizado independentemente de demanda). A simula��o durou sessenta segundos e foi realizada utilizando a ferramenta Pylot \cite{ref-pylot}.

O desempenho da solu��o utilizando banco de dados foi cerca de cinco vezes superior � solu��o com atualiza��o sob demanda como mostra a tabela~\ref{Tab:comparacao_de_solucoes}.

\begin{table}[htbp]
\begin{tabularx}{\linewidth}{lccc} \hline
Solu��o & Tempo m�dio (s) & \textit{Throughput} & Requisi��es \\ \hline
Utilizando banco de dados & 1,13 & 0,88 & 53 \\ \hline
Atualiza��o sob demanda & 5,53 & 0,17 & 10 \\ \hline
\end{tabularx}
\caption{Tabela com os resultados dos testes de performance das solu��es}
\label{Tab:comparacao_de_solucoes}
\end{table}

\begin{figure}[htbp!] \begin{center}
% fbox faz uma borda ao redor do seu argumento
\fbox{\includegraphics[width=0.97\linewidth]{img/grafico_tempos_de_resposta}}
\caption{Gr�fico comparativo de tempo m�dio de resposta em segundos}
\label{Fig:grafico_tempos_de_resposta}
\end{center} \end{figure}

\begin{figure}[htbp!] \begin{center}
% fbox faz uma borda ao redor do seu argumento
\fbox{\includegraphics[width=0.97\linewidth]{img/grafico_throughput}}
\caption{Gr�fico comparativo do throughput (n�mero de usu�rios atendidos por segundo)}
\label{Fig:grafico_throughput}
\end{center} \end{figure}

\section{Teste de desempenho do atualizador de cota��es}

O objetivo do teste de desempenho do atualizador foi analisar o impacto da varia��o do n�mero de processos no desempenho da atualiza��o das cota��es.

Foram aferidos os tempos de atualiza��o de 151 cota��es variando o n�mero de processos entre 1 e 16. Para o caso com um processo, a solu��o se assemelha muito a uma solu��o sem multiprocessamento. � medida que o n�mero de processos aumenta, o tempo necess�rio para atualizar todos os dados diminui at� o ponto em que o tempo de escalonamento de processos come�a a ficar significativo. Um ponto importante a ser observado � que o tempo de processamento efetivo � muito inferior ao tempo total, o que indica que os processadores passam muito tempo ociosos aguardando a resposta da fonte de dados externa. 

\begin{table}[htbp]
\begin{tabularx}{\linewidth}{lccc} \hline
No. processos & Tempo total (s) & Tempo de usu�rio (s) & Tempo de sistema (s) \\ \hline
1 & 138,3 & 1,01 & 0,22 \\ \hline
2 & 76,57 & 1,26 & 0,28 \\ \hline
3 & 50,61 & 1,41 & 0,27 \\ \hline
4 & 40,01 & 1,57 & 0,30 \\ \hline
5 & 33,02 & 1,71 & 0,33 \\ \hline
6 & 26,48 & 1,75 & 0,34 \\ \hline
7 & 24,31 & 1,92 & 0,36 \\ \hline
8 & 20,33 & 1,99 & 0,37 \\ \hline
9 & 19,87 & 2,15 & 0,40 \\ \hline
10 & 16,71 & 2,25 & 0,42 \\ \hline
11 & 15,28 & 2,42 & 0,44 \\ \hline
12 & 15,60 & 2,54 & 0,46 \\ \hline
13 & 11,80 & 2,63 & 0,48 \\ \hline
14 & 11,73 & 2,74 & 0,51 \\ \hline
15 & 11,19 & 2,85 & 0,53 \\ \hline
16 & 11,61 & 2,96 & 0,54 \\ \hline
\end{tabularx}
\caption{Tabela com os tempos aferidos de acordo com o n�mero de processos}
\label{Tab:tempos_aferidos}
\end{table}

\begin{figure}[htbp!] \begin{center}
% fbox faz uma borda ao redor do seu argumento
\fbox{\includegraphics[width=0.97\linewidth]{img/grafico_tempos_aferidos}}
\caption{Gr�fico de tempos aferidos de acordo com o n�mero de processos}
\label{Fig:grafico_tempos_aferidos}
\end{center} \end{figure}

%% \mychapter{Conclus�o}
\label{Cap:conclusao}

Com este trabalho � poss�vel concluir que a integra��o entre sistemas distintos � uma �rea muito desafiadora, principalmente quando envolve requisitos de atualiza��o em tempo real. Muitas s�o as possibilidades de arquitetura para otimizar a integra��o, cabe ao respons�vel pelo projeto analisar o problema e selecionar a melhor op��o.

A escolha desta op��o n�o deve ser apenas te�rica. Testes de desempenho s�o muito importantes e provas de conceito se fazem essenciais.



% Referncias bibliogficas (geradas automaticamente)
\addcontentsline{toc}{chapter}{Referências bibliográficas}
\bibliography{bibliografia}

%\appendix

%Apndice A 
% \include{apendice/apendice}

\end{document}
