\mychapter{Implementação}
\label{Cap:implementacao}

A implementação foi feita utilizando a linguagem de programação Python. Foi escolhido um cliente para o protocolo SOAP e encontrada uma API que disponibiliza cotações de moedas através de um \textit{web service} que utiliza este protocolo.

\section{Por que Python?}

Python é uma linguagem interpretada, dinâmica e fortemente tipada. Por ter estas características torna o trabalho mais rápido e diminui o tempo de desenvolvimento do projeto, que no caso é apenas uma prova de conceito, portanto o tempo curto de desenvolvimento é bastante interessante. Isso não significa que Python seja uma linguagem apropriada apenas para provas de conceito, visto que grandes empresas adotam esta teconologia em seus projetos que estão hoje em produção.

\section{Cliente SOAP}

\section{Estratégias de utilização do cliente}

Foram desenvolvidas duas implementações para o problema com o objetivo de encontrar a mais performática:

\begin{itemize}
\item atualização de dados sob demanda;
\item atualizador de dados independente de demanda.
\end{itemize}

\subsection{Requisições sob demanda}



\subsection{Atualizar todos os dados de tempos em tempos}

